\documentclass[11pt]{article}
\usepackage{amsmath,amssymb,amsthm}
\usepackage{mathtools}
\usepackage{geometry}
\geometry{margin=1in}

\newcommand{\KL}{D_{\mathrm{KL}}}
\newcommand{\E}{\mathbb{E}}
\newcommand{\R}{\mathbb{R}}

\title{KL Divergence Asymmetry for Exponential Families}
\author{}
\date{}

\begin{document}
\maketitle

\section{Setup}

Consider an exponential family distribution:
\begin{equation}
p(x \mid \theta) = h(x) \exp\bigl[\theta^T T(x) - A(\theta)\bigr]
\end{equation}
where:
\begin{itemize}
    \item $\theta \in \R^k$ are the \textbf{natural parameters}
    \item $T(x) \in \R^k$ are the \textbf{sufficient statistics}
    \item $A(\theta)$ is the \textbf{log-partition function} (cumulant generating function)
    \item $h(x)$ is the base measure
\end{itemize}

The \textbf{mean parameters} are:
\begin{equation}
\eta = \nabla A(\theta) = \E_\theta[T(x)]
\end{equation}

The \textbf{Fisher information} equals the Hessian of the log-partition:
\begin{equation}
\mathcal{I}(\theta) = \nabla^2 A(\theta) = \mathrm{Cov}_\theta[T(x)]
\end{equation}

\section{KL Divergence as Bregman Divergence}

For exponential families, the KL divergence admits a Bregman divergence representation. Let $p_\theta$ and $p_\phi$ be two distributions in the family with natural parameters $\theta$ and $\phi$ respectively.

\begin{align}
\KL(p_\theta \| p_\phi) &= \int p_\theta(x) \log \frac{p_\theta(x)}{p_\phi(x)} \, dx \\
&= \int p_\theta(x) \bigl[(\theta - \phi)^T T(x) - A(\theta) + A(\phi)\bigr] dx \\
&= (\theta - \phi)^T \E_\theta[T(x)] + A(\phi) - A(\theta) \\
&= A(\phi) - A(\theta) - \eta^T(\phi - \theta)
\end{align}
where $\eta = \nabla A(\theta)$.

This is the \textbf{Bregman divergence} $D_A(\phi \| \theta)$ generated by the convex function $A$:
\begin{equation}
\boxed{\KL(p_\theta \| p_\phi) = D_A(\phi \| \theta) = A(\phi) - A(\theta) - \nabla A(\theta)^T(\phi - \theta)}
\end{equation}

\section{Evaluating the Asymmetry}

Let $q_i$ have natural parameter $\theta$ and $Mq_i$ have natural parameter $\phi$. Define:
\begin{align}
\eta &= \nabla A(\theta) \quad \text{(mean parameters of } q_i\text{)} \\
\mu &= \nabla A(\phi) \quad \text{(mean parameters of } Mq_i\text{)} \\
\Delta\theta &= \phi - \theta \quad \text{(parameter difference)}
\end{align}

\subsection{Forward and Reverse KL}

\textbf{Forward KL:}
\begin{equation}
\KL(q_i \| Mq_i) = A(\phi) - A(\theta) - \eta^T(\phi - \theta) = A(\phi) - A(\theta) - \eta^T \Delta\theta
\end{equation}

\textbf{Reverse KL:}
\begin{align}
\KL(Mq_i \| q_i) &= A(\theta) - A(\phi) - \mu^T(\theta - \phi) \\
&= A(\theta) - A(\phi) + \mu^T \Delta\theta
\end{align}

\subsection{Exact Expression for the Difference}

Taking the difference:
\begin{align}
\KL(q_i \| Mq_i) - \KL(Mq_i \| q_i) &= \bigl[A(\phi) - A(\theta) - \eta^T \Delta\theta\bigr] - \bigl[A(\theta) - A(\phi) + \mu^T \Delta\theta\bigr] \\
&= 2\bigl[A(\phi) - A(\theta)\bigr] - \eta^T \Delta\theta - \mu^T \Delta\theta
\end{align}

\begin{equation}
\boxed{\KL(q_i \| Mq_i) - \KL(Mq_i \| q_i) = 2\bigl[A(\phi) - A(\theta)\bigr] - (\eta + \mu)^T \Delta\theta}
\end{equation}

For comparison, the \textbf{sum} takes the simpler form:
\begin{equation}
\KL(q_i \| Mq_i) + \KL(Mq_i \| q_i) = (\mu - \eta)^T \Delta\theta = \Delta\eta^T \Delta\theta
\end{equation}

\section{Taylor Expansion at the Midpoint}

To understand the structure, expand around the midpoint $\bar\theta = \tfrac{1}{2}(\theta + \phi)$.

\subsection{Expansion of the Log-Partition Function}

Let $\delta = \Delta\theta/2$, so $\phi = \bar\theta + \delta$ and $\theta = \bar\theta - \delta$.

\begin{align}
A(\phi) &= A(\bar\theta) + \partial_i A \, \delta^i + \frac{1}{2}\partial_i\partial_j A \, \delta^i\delta^j + \frac{1}{6}\partial_i\partial_j\partial_k A \, \delta^i\delta^j\delta^k + O(\delta^4) \\
A(\theta) &= A(\bar\theta) - \partial_i A \, \delta^i + \frac{1}{2}\partial_i\partial_j A \, \delta^i\delta^j - \frac{1}{6}\partial_i\partial_j\partial_k A \, \delta^i\delta^j\delta^k + O(\delta^4)
\end{align}
where all derivatives are evaluated at $\bar\theta$.

Therefore:
\begin{equation}
A(\phi) - A(\theta) = 2\partial_i A \, \delta^i + \frac{1}{3}\partial_i\partial_j\partial_k A \, \delta^i\delta^j\delta^k + O(\delta^4)
\end{equation}

In terms of $\Delta\theta = 2\delta$:
\begin{equation}
A(\phi) - A(\theta) = \nabla A(\bar\theta)^T \Delta\theta + \frac{1}{24}\partial_i\partial_j\partial_k A \, \Delta\theta^i \Delta\theta^j \Delta\theta^k + O(|\Delta\theta|^4)
\end{equation}

\subsection{Expansion of Mean Parameters}

Similarly:
\begin{align}
\eta_i = \partial_i A(\theta) &= \partial_i A(\bar\theta) - \partial_i\partial_j A \, \delta^j + \frac{1}{2}\partial_i\partial_j\partial_k A \, \delta^j\delta^k + O(\delta^3) \\
\mu_i = \partial_i A(\phi) &= \partial_i A(\bar\theta) + \partial_i\partial_j A \, \delta^j + \frac{1}{2}\partial_i\partial_j\partial_k A \, \delta^j\delta^k + O(\delta^3)
\end{align}

Adding:
\begin{equation}
\eta_i + \mu_i = 2\partial_i A(\bar\theta) + \partial_i\partial_j\partial_k A \, \delta^j\delta^k + O(\delta^3)
\end{equation}

Contracting with $\Delta\theta$:
\begin{equation}
(\eta + \mu)^T \Delta\theta = 2\nabla A(\bar\theta)^T \Delta\theta + \frac{1}{4}\partial_i\partial_j\partial_k A \, \Delta\theta^i\Delta\theta^j\Delta\theta^k + O(|\Delta\theta|^4)
\end{equation}

\subsection{The Asymmetry}

Substituting into the exact expression:
\begin{align}
\KL(q_i \| Mq_i) - \KL(Mq_i \| q_i) &= 2\bigl[A(\phi) - A(\theta)\bigr] - (\eta + \mu)^T \Delta\theta \\
&= 2\nabla A(\bar\theta)^T \Delta\theta + \frac{1}{12}\partial_i\partial_j\partial_k A \, \Delta\theta^i\Delta\theta^j\Delta\theta^k \\
&\quad - 2\nabla A(\bar\theta)^T \Delta\theta - \frac{1}{4}\partial_i\partial_j\partial_k A \, \Delta\theta^i\Delta\theta^j\Delta\theta^k + O(|\Delta\theta|^4) \\
&= \Bigl(\frac{1}{12} - \frac{1}{4}\Bigr)\partial_i\partial_j\partial_k A \, \Delta\theta^i\Delta\theta^j\Delta\theta^k + O(|\Delta\theta|^4) \\
&= -\frac{1}{6}\partial_i\partial_j\partial_k A \, \Delta\theta^i\Delta\theta^j\Delta\theta^k + O(|\Delta\theta|^4)
\end{align}

\begin{equation}
\boxed{\KL(q_i \| Mq_i) - \KL(Mq_i \| q_i) = -\frac{1}{6}\frac{\partial^3 A}{\partial\theta^i\partial\theta^j\partial\theta^k}\bigg|_{\bar\theta} \Delta\theta^i \Delta\theta^j \Delta\theta^k + O(|\Delta\theta|^4)}
\end{equation}

\section{Interpretation via Cumulants}

The third derivative of the log-partition function is the \textbf{third cumulant tensor}:
\begin{equation}
\frac{\partial^3 A}{\partial\theta^i\partial\theta^j\partial\theta^k} = \kappa^{ijk} = \E\bigl[(T^i - \eta^i)(T^j - \eta^j)(T^k - \eta^k)\bigr]
\end{equation}

This is the \textbf{skewness tensor} of the sufficient statistics. The KL asymmetry becomes:

\begin{equation}
\boxed{\KL(q_i \| Mq_i) - \KL(Mq_i \| q_i) = -\frac{1}{6}\kappa^{ijk}(\bar\theta)\,\Delta\theta_i\Delta\theta_j\Delta\theta_k + O(|\Delta\theta|^4)}
\end{equation}

\textbf{Key insight:} The KL asymmetry vanishes to second order for \emph{all} exponential families. The leading contribution is \textbf{third-order} and proportional to the \textbf{skewness} of the family.

\section{Specific Families}

\subsection{Gaussian Family}

For Gaussians, all cumulants beyond second order vanish: $\kappa^{ijk} = 0$.

\textbf{Equal covariances:} The asymmetry is \textbf{exactly zero}:
\begin{equation}
\KL\bigl(\mathcal{N}(\mu_1, \Sigma) \| \mathcal{N}(\mu_2, \Sigma)\bigr) = \KL\bigl(\mathcal{N}(\mu_2, \Sigma) \| \mathcal{N}(\mu_1, \Sigma)\bigr) = \frac{1}{2}(\mu_1 - \mu_2)^T\Sigma^{-1}(\mu_1 - \mu_2)
\end{equation}

\textbf{Different covariances:} For univariate Gaussians $\mathcal{N}_1 = \mathcal{N}(\mu_1, \sigma_1^2)$ and $\mathcal{N}_2 = \mathcal{N}(\mu_2, \sigma_2^2)$:
\begin{align}
\KL(\mathcal{N}_1 \| \mathcal{N}_2) &= \log\frac{\sigma_2}{\sigma_1} + \frac{\sigma_1^2 + (\mu_1 - \mu_2)^2}{2\sigma_2^2} - \frac{1}{2} \\
\KL(\mathcal{N}_2 \| \mathcal{N}_1) &= \log\frac{\sigma_1}{\sigma_2} + \frac{\sigma_2^2 + (\mu_1 - \mu_2)^2}{2\sigma_1^2} - \frac{1}{2}
\end{align}

The difference:
\begin{equation}
\boxed{\KL(\mathcal{N}_1 \| \mathcal{N}_2) - \KL(\mathcal{N}_2 \| \mathcal{N}_1) = 2\log\frac{\sigma_2}{\sigma_1} + \frac{(\sigma_1^2 - \sigma_2^2)\bigl[\sigma_1^2 + \sigma_2^2 + (\mu_1-\mu_2)^2\bigr]}{2\sigma_1^2\sigma_2^2}}
\end{equation}

\subsection{Exponential Distribution}

For $p(x|\lambda) = \lambda e^{-\lambda x}$ with natural parameter $\theta = -\lambda$:
\begin{equation}
A(\theta) = -\log(-\theta), \quad \frac{\partial^3 A}{\partial\theta^3} = \frac{2}{\theta^3} \neq 0
\end{equation}

The asymmetry is non-zero and scales as $O(|\Delta\theta|^3)$.

\subsection{Poisson Distribution}

For $p(k|\lambda) = \frac{\lambda^k e^{-\lambda}}{k!}$ with natural parameter $\theta = \log\lambda$:
\begin{equation}
A(\theta) = e^\theta, \quad \frac{\partial^3 A}{\partial\theta^3} = e^\theta = \lambda \neq 0
\end{equation}

Again, non-zero third cumulant yields non-vanishing asymmetry.

\subsection{Gamma Distribution}

For $p(x|\alpha,\beta) \propto x^{\alpha-1}e^{-\beta x}$, the natural parameters are $\theta_1 = \alpha - 1$ and $\theta_2 = -\beta$. The log-partition function is:
\begin{equation}
A(\theta_1, \theta_2) = \log\Gamma(\theta_1 + 1) - (\theta_1 + 1)\log(-\theta_2)
\end{equation}

The third cumulant tensor involves polygamma functions and is generally non-zero.

\section{Geometric Interpretation}

The KL asymmetry measures the \textbf{curvature asymmetry} of the statistical manifold.

\begin{itemize}
\item \textbf{Second order:} Both forward and reverse KL share the same local quadratic approximation---the \textbf{Fisher-Rao metric}. Hence no asymmetry at this order.

\item \textbf{Third order:} The manifold curves differently in opposite directions. This is captured by the \textbf{Amari-Chentsov tensor} (related to the third cumulant), which measures the deviation from a dually-flat geometry.
\end{itemize}

In information geometry, the \textbf{$\alpha$-connections} interpolate between the exponential (e-) and mixture (m-) connections. The KL asymmetry reflects the non-vanishing of the cubic term in the $\alpha$-geodesic expansion.

\section{Summary}

\begin{center}
\renewcommand{\arraystretch}{1.3}
\begin{tabular}{|l|c|c|}
\hline
\textbf{Family} & \textbf{Third Cumulant} & \textbf{KL Asymmetry} \\
\hline
Gaussian (fixed $\Sigma$) & $\kappa^{ijk} = 0$ & Exactly zero \\
Gaussian (varying $\Sigma$) & via $\Sigma$ & Non-zero \\
Poisson & $\kappa = \lambda$ & $O(|\Delta\theta|^3)$ \\
Exponential & $\kappa = 2/\lambda^3$ & $O(|\Delta\theta|^3)$ \\
Gamma & Polygamma terms & $O(|\Delta\theta|^3)$ \\
Categorical & Simplex geometry & $O(|\Delta\theta|^3)$ \\
\hline
\end{tabular}
\end{center}

\vspace{1em}

The key results:
\begin{enumerate}
\item The KL asymmetry \textbf{vanishes to second order} for all exponential families.
\item The \textbf{leading contribution is third-order}, proportional to the skewness tensor:
\begin{equation}
\KL(q_i \| Mq_i) - \KL(Mq_i \| q_i) = -\frac{1}{6}\kappa^{ijk}(\bar\theta)\,\Delta\theta_i\Delta\theta_j\Delta\theta_k + O(|\Delta\theta|^4)
\end{equation}
\item For \textbf{Gaussians with matched covariance}, the asymmetry is exactly zero.
\item The asymmetry measures the \textbf{curvature asymmetry} of the statistical manifold.
\end{enumerate}

\end{document}
