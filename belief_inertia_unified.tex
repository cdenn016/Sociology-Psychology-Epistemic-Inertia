\documentclass[12pt]{article}
\usepackage{amsmath,amssymb,amsthm}
\usepackage{geometry}
\usepackage[round]{natbib}
\usepackage{graphicx}
\usepackage{float}
\usepackage{booktabs}
\usepackage{array}
\usepackage{tcolorbox}
\usepackage{subcaption}
\usepackage{hyperref}
\usepackage{enumitem}
\geometry{margin=1in}
\newcommand{\KL}{\mathrm{KL}}
\newcommand{\tr}{\mathrm{tr}}
\newcommand{\Sig}{\Sigma}
\newcommand{\SigQ}{\Sigma^q}
\newcommand{\SigP}{\Sigma^p}
\newcommand{\muQ}{\mu^q}
\newcommand{\muP}{\mu^p}
\newcommand{\Dmu}{\Delta\mu}
\newcommand{\vech}{\mathrm{vech}}
\newcommand{\Prec}{P}
\newcommand{\R}{\mathbb{R}}
\newcommand{\N}{\mathcal{N}}
\newtheorem{theorem}{Theorem}
\newtheorem{proposition}[theorem]{Proposition}
\newtheorem{definition}[theorem]{Definition}
\newtheorem{prediction}[theorem]{Prediction}
\newtheorem{lemma}[theorem]{Lemma}
\newtheorem{corollary}[theorem]{Corollary}

\title{The Inertia of Belief: A Unified Framework for Cognitive and Social Dynamics}

\author{
Robert C. Dennis\\
\texttt{cdenn016@gmail.com}
}

\date{\today}

\begin{document}
\maketitle


\begin{abstract}
We present a unified theoretical framework demonstrating that variational free energy minimization on statistical manifolds produces Hamiltonian mechanics for beliefs, with the Fisher information metric emerging as an inertial mass tensor. This framework not only explains individual cognitive phenomena---including belief oscillation, overshooting, and epistemic resonance---but also unifies major models from sociology, psychology, and network science as limiting cases. We derive rigorous proofs showing that DeGroot social learning, Friedkin-Johnsen opinion dynamics, bounded confidence models, confirmation bias, Social Impact Theory, echo chamber formation, diffusion of innovations, collective intentionality, and Kuhnian paradigm shifts all emerge from a single information-geometric framework under different parameter regimes. The key insight is that precision (inverse uncertainty) functions as epistemic mass: confident beliefs resist change while uncertain beliefs update readily. Interactions between agents are encoded via a gauge-invariant attention mechanism enabling communication across reference frames. Standard Bayesian updating emerges as the overdamped limit; in underdamped regimes, the framework predicts belief oscillation, overshooting, and epistemic resonance. The framework offers a geometric explanation for confirmation bias, belief perseverance, and opinion polarization as epistemic inertia rather than irrationality, while generating novel testable predictions that distinguish it from existing theories.
\end{abstract}


\noindent\textbf{Keywords:} Gauge theory $\cdot$ Active inference $\cdot$ Free energy principle $\cdot$ Information geometry $\cdot$ Sociology $\cdot$ Opinion dynamics

\tableofcontents

\section{Introduction}

Why do some beliefs resist change more than others? Why does influence itself seem to harden the minds of those who wield it? Some beliefs are stiff while others readily sway. While confident beliefs clearly possess more ``cognitive inertia'' than uncertain ones, a principled mathematical foundation for this intuitive phenomenon remains elusive. Current theories of belief updating, from Bayesian inference \citep{jaynes2003probability} to predictive coding \citep{friston2010,clark2013whatever}, model belief change as gradient descent---a purely dissipative process where beliefs flow toward lower free energy without momentum, inertia, or dynamics. Though enormously successful across neuroscience \citep{friston2016active}, psychology \citep{hohwy2013predictive}, and machine learning \citep{millidge2021predictive}, this framework fundamentally remains incomplete.

In this article, we show that beliefs possess an ``epistemic'' inertia proportional to an agent's prior precision. Just as physical objects with mass resist acceleration, beliefs held with high confidence (precision) resist change and, once moving, tend to continue in their direction. This is not merely metaphor but a mathematical consequence of a second-order expansion of the variational free energy. The Fisher information metric \citep{Amari2016}, which measures statistical distinguishability, simultaneously provides an inertial mass tensor for belief dynamics.

Furthermore, beliefs propagate through networks of agents in attention patterns ranging from coordinated consensus to turbulent disagreement, often exhibiting distortion, resonance, and phase transitions \citep{castellano2009statistical,galam2012sociophysics}. Social influence research has produced numerous models describing how individuals update beliefs through social interaction---from DeGroot's social learning \citep{degroot1974} to Friedkin-Johnsen opinion dynamics \citep{friedkin2011}, bounded confidence models \citep{hegselmann2002opinion}, and Social Impact Theory \citep{latane1981}. While empirically successful in their respective domains, these models appear theoretically disconnected.

We demonstrate that this fragmentation is illusory: major classical models emerge as limiting cases of a unified framework based on variational free energy minimization on statistical manifolds. This unification proceeds by analogy to statistical mechanics. Just as thermodynamic phases (solid, liquid, gas) emerge from molecular dynamics under different conditions (temperature, pressure), we show that classical social influence models emerge from information-geometric dynamics under different conditions (friction, uncertainty, attention temperature).

Our framework provides three primary contributions:

\begin{enumerate}

\item \textbf{Theoretical Foundation}: We derive second-order belief dynamics from first principles, showing that the Fisher metric provides a natural inertial mass tensor $M = \Sigma_p^{-1} = \Lambda_p$ (prior precision). Via pullback geometry on informational bundles, we extend variational free energy to multi-agent systems characterized by belief-momentum exchange and gauge-covariant transport.

\item \textbf{Unification of Classical Models}: We provide rigorous derivations showing that DeGroot social learning, Friedkin-Johnsen dynamics, bounded confidence, echo chambers, confirmation bias, Social Impact Theory, diffusion of innovations, collective intentionality, and Kuhnian paradigm shifts all emerge as special or limiting cases of this framework.

\item \textbf{Explanatory Power}: We unify documented but theoretically orphaned phenomena---attitude oscillation in persuasion \citep{kaplowitz1992,fink2002}, perceptual overshoot \citep{burge2010,webster2015}, momentum in economic expectations \citep{coibion2015}, confirmation bias \citep{nickerson1998confirmation}, and belief perseverance \citep{anderson1980perseverance}---as natural consequences of belief inertia operating in different parameter regimes.

\end{enumerate}

\section{Mathematical Framework}

\subsection{Beliefs as Points on Statistical Manifolds}

We model beliefs as probability distributions $q(\theta)$ parameterized by $\theta \in \mathbb{R}^n$ on a statistical manifold $\mathcal{M}$. For the remainder of this article we consider multivariate Gaussian (MVG) beliefs and priors:

\begin{align}
q &= \mathcal{N}(\mu_q, \Sigma_q) \\
p &= \mathcal{N}(\mu_p, \Sigma_p)
\end{align}

where $\mu_\nu$ represents the believed value and $\Sigma_\nu$ represents uncertainty.

The Kullback-Leibler (KL) divergence measures the epistemic distance between an agent's belief $q$ and their prior model $p$:

\begin{equation}
\text{KL}(q \| p) = \int q(x) \log \frac{q(x)}{p(x)} dx
\end{equation}

\subsection{The Variational Free Energy Functional}

Consider $N$ agents, each maintaining a belief distribution $q_i(x) = \N(\mu_i, \Sigma_i)$ and prior $p_i(x) = \N(\mu_{p,i}, \Sigma_{p,i})$ over a latent state $x \in \R^K$. The total variational free energy is:

\begin{align}
F[q, p] &= \sum_i \alpha \int \chi_i(c) \KL(q_i \| p_i) \, dc \label{eq:vfe_self} \\
&\quad + \sum_{i,j} \lambda_\beta \int \chi_{ij}(c) \beta_{ij}(c) \KL(q_i \| \Omega_{ij}[q_j]) \, dc \label{eq:vfe_belief} \\
&\quad + \sum_{i,j} \lambda_\gamma \int \chi_{ij}(c) \gamma_{ij}(c) \KL(p_i \| \Omega_{ij}[p_j]) \, dc \label{eq:vfe_prior} \\
&\quad - \sum_i \lambda_{\text{obs}} \int \chi_i(c) \mathbb{E}_q[\log p(o|x)] \, dc \label{eq:vfe_obs}
\end{align}

Each term serves a distinct functional role in belief dynamics:
\begin{itemize}
\item The self-coupling term \eqref{eq:vfe_self} penalizes deviation from prior expectations, with strength parameter $\alpha$ controlling resistance to belief change.
\item The belief alignment term \eqref{eq:vfe_belief} encourages agents to align current beliefs with neighbors, where attention weights $\beta_{ij}(c)$ are computed dynamically via softmax.
\item The prior alignment term \eqref{eq:vfe_prior} operates on priors themselves, capturing slower cultural or normative alignment.
\item The observation term \eqref{eq:vfe_obs} grounds beliefs in sensory evidence.
\end{itemize}

\subsection{Multi-Agent Belief Geometry}

We model agents as residing on a gauge-theoretic bundle geometry where each agent $i$ maintains beliefs $q_i = \mathcal{N}(\mu_i, \Sigma_i)$ and an internal reference frame $\phi_i$ determining how they interpret information.

Agents cannot directly compare beliefs. They must first align their gauge frames via parallel transport operators:

\begin{equation}
\Omega_{ij} = e^{\phi_i}e^{-\phi_j}
\end{equation}

This operator transforms agent $j$'s beliefs into agent $i$'s gauge frame of reference:

\begin{equation}
q_j \to \Omega_{ij} \cdot q_j = \mathcal{N}(\Omega_{ij}\mu_j, \Omega_{ij}\Sigma_j\Omega_{ij}^T)
\end{equation}

\subsection{Softmax Attention Mechanism}

The belief alignment weights are computed dynamically as:
\begin{equation}
\beta_{ij}(c) = \frac{\exp\left(-\KL(q_i(c) \| \Omega_{ij}[q_j(c)]) / \kappa_\beta\right)}{\sum_k \exp\left(-\KL(q_i(c) \| \Omega_{ik}[q_k(c)]) / \kappa_\beta\right)}
\label{eq:softmax_attention}
\end{equation}

This softmax creates \emph{homophilic attention}: agents with similar beliefs (low KL divergence) receive high attention, while dissimilar agents are ignored. The temperature parameter $\kappa_\beta > 0$ controls sharpness: $\kappa_\beta \to 0$ gives winner-take-all attention, while $\kappa_\beta \to \infty$ gives uniform attention.

\subsection{Fisher-Rao Mass Matrix (Epistemic Inertia)}

The natural gradient descent dynamics on the statistical manifold are governed by the inverse Fisher metric (mass matrix):
\begin{equation}
M_i(\theta) = \Sigma_{p,i}^{-1} + \Sigma_{o,i}^{-1} + \sum_j \beta_{ij}(\theta) \, \Omega_{ij} \Sigma_{q,j}^{-1} \Omega_{ij}^T + \sum_j \beta_{ji}(\theta) \, \Sigma_{q,i}^{-1}
\label{eq:mass_matrix}
\end{equation}

This matrix comprises four physically interpretable contributions:
\begin{itemize}
\item $\Sigma_{p,i}^{-1}$: \emph{Prior mass}---resistance from established expectations
\item $\Sigma_{o,i}^{-1}$: \emph{Observation mass}---anchoring from precise sensory evidence
\item $\sum_j \beta_{ij} \Omega_{ij} \Sigma_{q,j}^{-1} \Omega_{ij}^T$: \emph{Incoming social mass}---inertia from attending to confident neighbors
\item $\sum_j \beta_{ji} \Sigma_{q,i}^{-1}$: \emph{Outgoing social mass}---recoil from exerting influence on others
\end{itemize}

\subsection{Dynamics: Overdamped vs. Hamiltonian Regimes}

The framework admits two dynamical regimes:

\paragraph{Overdamped (gradient flow):} When friction $\gamma \to \infty$, momentum dissipates instantly:
\begin{equation}
\frac{d\mu_i}{dt} = -M_i^{-1} \nabla_{\mu_i} F
\label{eq:overdamped}
\end{equation}

This is \emph{natural gradient descent} on the statistical manifold.

\paragraph{Hamiltonian (underdamped):} When friction $\gamma \to 0$, the system exhibits inertial dynamics:
\begin{align}
\frac{d\mu_i}{dt} &= M_i^{-1} \pi_{\mu,i} \label{eq:hamiltonian_position} \\
\frac{d\pi_{\mu,i}}{dt} &= -\nabla_{\mu_i} F - \Gamma_{ijk} \pi^j \pi^k - \gamma \pi_{\mu,i} \label{eq:hamiltonian_momentum}
\end{align}

Here $\pi_{\mu,i}$ is conjugate momentum, and $\Gamma_{ijk}$ are Christoffel symbols. This regime allows belief oscillations, overshooting, and non-monotonic convergence.

Classical models correspond to the \emph{overdamped regime} with various limiting conditions on parameters $(\alpha, \lambda_\beta, \kappa_\beta, \Sigma_i)$. The Hamiltonian regime generates novel predictions beyond classical theories.

\section{Cognitive Phenomena from Belief Momentum}
\label{sec:cognitive-momentum}

The Hamiltonian formulation introduces a quantity absent from standard Bayesian updating: epistemic momentum.

\begin{definition}[Cognitive Momentum]
The cognitive momentum of agent $i$ is the product of epistemic mass and belief velocity:
\begin{equation}
\boxed{\pi_i = M_i \dot{\mu}_i = \left(\bar{\Lambda}_{pi} + \Lambda_{oi}+ \sum_k \beta_{ik}\tilde{\Lambda}_{qk} + \sum_j \beta_{ji}\Lambda_{qi}\right) \dot{\mu}_i}
\end{equation}
\end{definition}

\subsection{Confirmation Bias as Momentum}

The stopping distance for a belief moving at velocity $\dot{\mu}$ against constant opposing force $f$ is:

\begin{equation}
d_{\text{stop}} = \frac{M_i \|\dot{\mu}_i\|^2}{2\|f\|}= \frac{\|\pi_i\|^2}{2M_i\|f\|}
\end{equation}

A person twice as confident takes twice as long to stop and overshoots twice as far. This reframes confirmation bias as epistemic inertia rather than irrationality.

\subsection{Belief Oscillation and Overshooting}

Including dissipation, the equation of motion becomes:

\begin{equation}
M_i\ddot{\mu}_i + \gamma_i\dot{\mu}_i + \nabla_{\mu_i}F = 0
\end{equation}

The discriminant $\Delta = \gamma_i^2 - 4K_iM_i$ determines three regimes:

\begin{enumerate}
\item \textbf{Overdamped} ($\Delta > 0$): Monotonic decay---standard Bayesian updating
\item \textbf{Critically damped} ($\Delta = 0$): Fastest equilibration without oscillation
\item \textbf{Underdamped} ($\Delta < 0$): Oscillatory dynamics with overshooting
\end{enumerate}

In the underdamped regime:
\begin{equation}
\boxed{\omega = \sqrt{\frac{K_i}{M_i} - \frac{\gamma_i^2}{4M_i^2}} \approx \sqrt{\frac{\text{Evidence strength}}{\text{Epistemic mass}}}}
\end{equation}

\subsection{Cognitive Resonance}

Periodic evidence achieves maximum belief change at the resonance frequency:
\begin{equation}
\boxed{\omega_{\text{res}} = \sqrt{\frac{K_i}{M_i}} = \sqrt{\frac{\text{Evidence strength} \times \text{Precision}}{\text{Epistemic mass}}}}
\end{equation}

High-mass (confident) agents have larger resonance amplitudes. While they resist off-resonance forcing, properly timed evidence produces dramatic swings.

\subsection{Belief Perseverance}

The characteristic relaxation time is:
\begin{equation}
\boxed{\tau = \frac{M_i}{\gamma_i} = \frac{\bar{\Lambda}_{pi} + \Lambda_{oi}+ \sum_k\beta_{ik}\tilde{\Lambda}_{qk} + \sum_j\beta_{ji}\Lambda_{qi}}{\gamma_i}}
\end{equation}

High-precision beliefs have long decay times, explaining why false beliefs persist even after debunking.

\subsection{Momentum Transfer Between Agents}

\begin{theorem}[Momentum Transfer Between Agents]
When agent $k$ changes belief, it transfers epistemic momentum to agent $i$ according to:
\begin{equation}
\frac{d\pi_i}{dt}\bigg|_{\text{from } k} = -\beta_{ik}\tilde{\Lambda}_{qk}(\mu_i - \tilde{\mu}_k) - \beta_{ki}\Lambda_{qi}\Omega_{ki}^T(\tilde{\mu}_k^{(i)} - \mu_i)
\end{equation}
\end{theorem}

Social influence has mechanical consequences: changing another's mind necessarily affects one's own epistemic trajectory.

%==============================================================================
\section{Classical Models as Limiting Cases}
\label{sec:classical_limits}
%==============================================================================

We now demonstrate that six major models from sociology, psychology, and network science emerge as special or limiting cases of the Hamiltonian Variational Free Energy framework.

\subsection{DeGroot Social Learning}

\subsubsection{Classical Formulation}

DeGroot's model (1974) describes social learning as iterative averaging of neighbors' beliefs:
\begin{equation}
x_i(t+1) = \sum_j w_{ij} x_j(t)
\label{eq:degroot_classical}
\end{equation}
where $W = [w_{ij}]$ is a row-stochastic matrix representing social influence weights.

\subsubsection{Derivation from VFE Framework}

\begin{proposition}[DeGroot as VFE Limit]
The DeGroot update rule emerges from the VFE framework under:
\begin{enumerate}[label=(\roman*)}
\item Overdamped dynamics: $\gamma \to \infty$
\item Low uncertainty: $\Sigma_i \to \sigma^2 I$ with $\sigma^2$ small
\item Flat manifold: $\Omega_{ij} = I$
\item No self-coupling: $\alpha = 0$
\item No prior alignment: $\lambda_\gamma = 0$
\item No observations: $\lambda_{\text{obs}} = 0$
\item Fixed attention: $\beta_{ij} = w_{ij}$ (constant)
\end{enumerate}
\end{proposition}

\begin{proof}
Under these conditions, the free energy reduces to:
\begin{equation}
F[\mu] = \frac{\lambda_\beta}{2\sigma^2} \sum_{i,j} w_{ij} \|\mu_i - \mu_j\|^2
\end{equation}

The gradient is:
\begin{equation}
\nabla_{\mu_i} F = \frac{\lambda_\beta}{\sigma^2} \left[\mu_i - \sum_j w_{ij} \mu_j\right]
\end{equation}

Natural gradient flow with mass $M_i \approx \sigma^{-2} I$ gives:
\begin{equation}
\frac{d\mu_i}{dt} = -\lambda_\beta \left(\mu_i - \sum_j w_{ij} \mu_j\right)
\end{equation}

Discretizing with $\Delta t = 1/\lambda_\beta$ yields exactly $\mu_i(t+1) = \sum_j w_{ij} \mu_j(t)$.
\end{proof}

\subsubsection{What the Unified Framework Adds}

\paragraph{Dynamic attention.} Using softmax attention, influence weights become endogenous:
\begin{equation}
\beta_{ij}(t) = \frac{\exp(-\|\mu_i(t) - \mu_j(t)\|^2 / (2\sigma^2\kappa_\beta))}{\sum_k \exp(-\|\mu_i(t) - \mu_k(t)\|^2 / (2\sigma^2\kappa_\beta))}
\end{equation}

\paragraph{Epistemic inertia.} High-attention agents (opinion leaders) develop higher mass via the $\sum_j \beta_{ji}$ term, making their beliefs more resistant to change.

\subsection{Friedkin-Johnsen Opinion Dynamics}

\subsubsection{Classical Formulation}

Friedkin and Johnsen (1990) extended DeGroot with ``stubbornness'':
\begin{equation}
x_i(t+1) = \alpha_i x_i(0) + (1 - \alpha_i) \sum_j w_{ij} x_j(t)
\label{eq:fj_classical}
\end{equation}
where $\alpha_i \in [0,1]$ represents resistance to social influence.

\subsubsection{Derivation from VFE Framework}

\begin{proposition}[Friedkin-Johnsen as VFE Equilibrium]
The Friedkin-Johnsen equilibrium emerges from the VFE framework with non-zero self-coupling $\alpha > 0$ and fixed priors $p_i = \N(\mu_i(0), \Sigma_p)$.
\end{proposition}

\begin{proof}
Including the self-coupling term:
\begin{equation}
F[\mu] = \frac{\alpha}{2\Sigma_p} \sum_i \|\mu_i - \mu_i(0)\|^2 + \frac{\lambda_\beta}{2\sigma^2} \sum_{i,j} w_{ij} \|\mu_i - \mu_j\|^2
\end{equation}

At steady state, $d\mu_i/dt = 0$ gives:
\begin{equation}
\mu_i = \alpha_i' \mu_i(0) + (1 - \alpha_i') \sum_j w_{ij} \mu_j
\end{equation}

where the \emph{emergent stubbornness} is:
\begin{equation}
\alpha_i' = \frac{\alpha}{\alpha + \lambda_\beta \Sigma_p / \sigma^2 \cdot \sum_j w_{ij}}
\end{equation}
\end{proof}

\paragraph{Key insight:} Stubbornness is not a fixed personality trait but emerges from prior precision and social context.

\subsection{Echo Chambers and Polarization}

\begin{proposition}[Emergent Homophily and Polarization]
Softmax attention automatically creates homophilic coupling. When initial belief distributions are multimodal, this leads to stable polarized states.
\end{proposition}

\begin{proof}[Proof Sketch]
For Gaussian beliefs with common covariance:
\begin{equation}
\beta_{ij} = \frac{\exp(-\|\mu_i - \mu_j\|^2 / (2\sigma^2 \kappa_\beta))}{\sum_k \exp(-\|\mu_i - \mu_k\|^2 / (2\sigma^2 \kappa_\beta))}
\end{equation}

This creates a positive feedback loop: similar beliefs $\to$ high attention $\to$ further convergence $\to$ cross-group attention vanishes.

The critical distance for polarization stability is:
\begin{equation}
\|\mu_A - \mu_B\|^2 > 2\sigma^2 \kappa_\beta \log N
\label{eq:polarization_threshold}
\end{equation}
\end{proof}

\paragraph{Phase transition.} The temperature $\kappa_\beta$ controls a phase transition:
\begin{itemize}
\item High temperature: diffuse attention, global consensus
\item Low temperature: sharp attention, polarized states
\end{itemize}

Social media platforms using engagement-based ranking effectively lower $\kappa_\beta$, increasing polarization.

\subsection{Bounded Confidence Models}

\subsubsection{Classical Formulation}

Hegselmann-Krause (2002) introduced bounded confidence:
\begin{equation}
x_i(t+1) = \text{avg}\{x_j(t) : |x_j(t) - x_i(t)| < \epsilon\}
\end{equation}

\subsubsection{Derivation from VFE Framework}

\begin{proposition}[Bounded Confidence as Low-Temperature Limit]
Bounded confidence approximates VFE in the low-temperature regime $\kappa_\beta \to 0$, with effective threshold:
\begin{equation}
\epsilon_{\text{eff}} \approx \sigma\sqrt{2\kappa_\beta \log N}
\end{equation}
\end{proposition}

As $\kappa_\beta \to 0$, the softmax becomes increasingly sharp, creating an effective radius beyond which attention decays exponentially to zero.

\paragraph{Key difference:} VFE produces a \emph{soft} threshold (smooth exponential decay) rather than the classical hard cutoff, which is more psychologically realistic and mathematically tractable.

\subsection{Confirmation Bias from Epistemic Inertia}

\begin{proposition}[Confirmation Bias from Epistemic Inertia]
Agents with high prior precision or many followers exhibit confirmation bias without any non-Bayesian mechanisms.
\end{proposition}

\begin{proof}[Proof Sketch]
The update dynamics are:
\begin{equation}
\frac{d\mu_i}{dt} = -M_i^{-1} \nabla_{\mu_i} F
\end{equation}

When mass $M_i$ is large (high prior precision or many followers), updates are small:
\begin{equation}
\Delta \mu_i \propto M_i^{-1} (\mu_i - \mu_{\text{evidence}})
\end{equation}

The outgoing social mass term $\sum_j \beta_{ji} \Sigma_{q,i}^{-1}$ means that \emph{influencing others costs flexibility}. Opinion leaders become trapped by their followers.
\end{proof}

This provides a geometric mechanism for what Adams called the ``poison'' of power: when many minds attend to yours, belief change becomes increasingly costly.

\subsection{Social Impact Theory}

Latan\'e's Social Impact Theory (1981) posits:
\begin{equation}
\text{Impact} = f(\text{Strength} \times \text{Immediacy} \times \text{Number})
\end{equation}

The mass matrix provides a natural quantitative interpretation:

\begin{itemize}
\item \textbf{Strength} $\leftrightarrow$ $\Sigma_{q,j}^{-1}$: Source precision (expertise)
\item \textbf{Immediacy} $\leftrightarrow$ $\|\Omega_{ij} - I\|$: Transport penalty (frame alignment)
\item \textbf{Number} $\leftrightarrow$ $\sum_j$: Count of sources
\end{itemize}

The VFE framework adds exact quantitative formulae, time-varying impact, and asymmetry (impact is not reciprocal).

\subsection{Diffusion of Innovations}

\begin{proposition}[S-Curve from Attention Dynamics]
The logistic adoption curve emerges when agents decide between adopt/reject based on social attention from prior adopters, with heterogeneous prior precisions determining adoption order.
\end{proposition}

The Rogers categories (innovators, early adopters, early majority, late majority, laggards) emerge naturally from the distribution of epistemic mass $M_i$:
\begin{itemize}
\item Innovators: low mass (uncertain priors, few anchoring connections)
\item Laggards: high mass (extreme prior certainty or isolation)
\end{itemize}

\paragraph{Commitment trap:} Early adopters who influence others accumulate social mass, becoming resistant to abandoning the innovation even if problems emerge.

\subsection{Collective Intentionality and We-Intentions}

\begin{proposition}[We-Mode as Phase Transition]
The transition from I-mode to we-mode may correspond to a phase transition when coupling strength exceeds a critical threshold and gauge frames align.
\end{proposition}

The framework suggests:
\begin{itemize}
\item \textbf{Gauge frame alignment}: We-mode corresponds to $\Omega_{ij} \approx I$ for in-group members
\item \textbf{Attention density transition}: From sparse (I-mode) to dense (we-mode)
\item \textbf{Prior convergence}: Shared generative models, not just aligned beliefs
\end{itemize}

\subsection{Kuhnian Paradigm Shifts}

Paradigms correspond to metastable states---local minima separated by barriers:
\begin{itemize}
\item \textbf{Normal science}: Overdamped relaxation within a basin
\item \textbf{Revolution}: Transition between basins requiring activation energy
\item \textbf{Incommensurability}: Gauge mismatch $\|\Omega_{AB} - I\|$ large
\end{itemize}

\paragraph{Planck's principle:} Senior scientists accumulate epistemic mass (high prior precision, many followers), creating barriers to paradigm change. ``Science advances one funeral at a time'' emerges from geometry, not irrationality.

%==============================================================================
\section{Summary of Derivations}
%==============================================================================

\begin{table}[h]
\centering
\begin{tabular}{@{}lccp{5.5cm}@{}}
\toprule
\textbf{Model} & \textbf{Rigor} & \textbf{Status} & \textbf{Notes} \\
\midrule
DeGroot & $\checkmark\checkmark\checkmark$ & Rigorous & Exact limit with proof \\
Friedkin-Johnsen & $\checkmark\checkmark\checkmark$ & Rigorous & Emergent stubbornness \\
Echo Chambers & $\checkmark\checkmark\checkmark$ & Rigorous & Direct from softmax \\
Bounded Confidence & $\checkmark\checkmark$ & Solid & Soft approximation \\
Confirmation Bias & $\checkmark\checkmark$ & Solid & Requires natural gradient \\
Social Impact Theory & $\checkmark\checkmark$ & Interpretive & Qualitative correspondence \\
Diffusion of Innovations & $\checkmark\checkmark$ & Solid & S-curve from attention \\
Collective Intentionality & $\checkmark$ & Speculative & Phase transition interpretation \\
Kuhnian Paradigm Shifts & $\checkmark\checkmark$ & Solid/Interp. & Metastable basins \\
\bottomrule
\end{tabular}
\caption{Quality assessment of derivations.}
\label{tab:rigor}
\end{table}

%==============================================================================
\section{Novel Predictions}
%==============================================================================

Beyond recovering classical models, the VFE framework generates novel empirical predictions:

\paragraph{Attention and network dynamics.}
\begin{itemize}
\item \emph{Dynamic attention}: Influence networks restructure as beliefs converge/diverge
\item \emph{Epistemic inertia}: High-attention agents update more slowly
\item \emph{Commitment trap}: Early adopters who promote innovations abandon them more slowly
\end{itemize}

\paragraph{Phase transitions.}
\begin{itemize}
\item Polarization emerges sharply at critical $\kappa_\beta^{\text{crit}}$
\item I-mode to we-mode transition exhibits hysteresis and critical slowing
\item Paradigm shifts display asymmetric hysteresis
\end{itemize}

\paragraph{Context-dependence.}
\begin{itemize}
\item Same individual exhibits different stubbornness across contexts
\item Adopter categories are domain-specific, not personality traits
\item Scientists' paradigm resistance correlates with field-specific mass
\end{itemize}

\paragraph{Geometric structure of communication.}
\begin{itemize}
\item Communication difficulty scales with frame rotation $\|\Omega_{ij} - I\|$
\item ``Talking past each other'' reflects gauge mismatch, not belief difference
\end{itemize}

\paragraph{Testable predictions.}
\begin{itemize}
\item Belief relaxation times scale with prior precision: $\tau = M/\gamma$
\item Confident beliefs overshoot equilibria when confronted with opposing evidence
\item Periodic persuasion peaks at resonance frequency $\omega = \sqrt{K/M}$
\item Attention spikes produce temporarily elevated confirmation bias
\end{itemize}

%==============================================================================
\section{Extended Framework: Meta-Agents and Hierarchical Dynamics}
%==============================================================================

\subsection{Meta-Agent Formation}

When agents achieve sufficient belief coherence, they can form \emph{meta-agents}: higher-order collective entities with their own beliefs, priors, and gauge frames.

\begin{definition}[Meta-Agent Formation Condition]
A cluster of scale-$s$ agents forms a scale-$(s+1)$ meta-agent when:
\begin{equation}
C_{\text{belief}} \cdot C_{\text{model}} > \Gamma_{\min}
\end{equation}
where $C_{\text{belief}} = \exp\left(-\frac{1}{|S|}\sum_{i,j \in S} \text{KL}(q_i \| \Omega_{ij}[q_j])\right)$.
\end{definition}

This formalizes the emergence of institutions, organizations, and collective actors.

\subsection{Dynamical Renormalization}

The free energy at scale $(s+1)$ has the same form as at scale $s$:
\begin{equation}
F^{(s+1)} = \sum_I \text{KL}(q_I^{(s+1)} \| p_I^{(s+1)}) + \sum_{IJ} \beta_{IJ}^{(s+1)} \text{KL}(q_I \| \Omega_{IJ}[q_J]) + \cdots
\end{equation}

This scale invariance suggests the same dynamical principles govern individuals, groups, organizations, and societies.

\paragraph{Emergent timescale hierarchy:}
\begin{equation}
\tau^{(0)} < \tau^{(1)} < \tau^{(2)} < \cdots < \tau^{(s_{\max})}
\end{equation}

Institutional change is \emph{necessarily} slower than individual opinion change---not due to bureaucratic friction but due to geometric structure.

\subsection{Bidirectional Information Flow}

\paragraph{Bottom-up (emergence):} Individual beliefs aggregate into collective beliefs through gauge-covariant averaging.

\paragraph{Top-down (constraint):} Meta-agent beliefs become priors for constituents: $p_i \leftarrow \Omega_{i,I}[q_I]$.

A functioning participatory system must remain far from equilibrium. Complete consensus (epistemic death) would freeze evolution.

%==============================================================================
\section{Discussion}
%==============================================================================

\subsection{Unifying Phenomenological Models}

For decades, researchers across psychology, neuroscience, and opinion dynamics have empirically modeled belief change using spring-mass metaphors without theoretical justification. Kaplowitz and Fink's damped oscillator model of attitude change, bounded confidence models in opinion dynamics, and momentum effects in economic expectations all invoke inertial dynamics without explaining their origin.

Our central contribution is showing that these are not mere analogies but consequences of variational inference on curved statistical manifolds. The Fisher information metric \textit{is} the inertial mass tensor; damping emerges from dissipative terms; the restoring force is the gradient of variational free energy.

\subsection{Cognitive Biases as Emergent Phenomena}

Phenomena often attributed to ``cognitive biases'' emerge naturally from epistemic inertia:

\begin{itemize}
\item \textbf{Belief perseverance}: High-precision beliefs have large mass and long relaxation times
\item \textbf{Continued influence effect}: Momentum decay rather than memory failure
\item \textbf{Confirmation bias}: Geometrical consequence of how massive objects respond to forces
\item \textbf{Rigidity of influence}: Outgoing social mass accumulates with followers
\end{itemize}

This reframing provides a mechanistic basis for intervention. If belief perseverance stems from high precision rather than stubbornness, then interventions targeting uncertainty may be more effective than those targeting content.

\subsection{Relation to Existing Models}

\begin{table}[ht]
\centering
\caption{Predictions distinguishing the inertial framework from first-order models.}
\begin{tabular}{lcc}
\hline
\textbf{Phenomenon} & \textbf{First-Order Models} & \textbf{This Framework} \\
\hline
Approach to equilibrium & Monotonic & Can oscillate \\
Precision dependence & Weights evidence & Determines inertia \\
Overshooting & Not predicted & Predicted (underdamped) \\
Resonance to periodic input & Not predicted & Predicted \\
Belief perseverance & Separate bias & Emerges from mass \\
Social momentum transfer & Absent & Predicted \\
Attention increases rigidity & Not predicted & Predicted (social mass) \\
\hline
\end{tabular}
\end{table}

\subsection{Limitations}

\begin{itemize}
\item \textbf{Gaussian beliefs}: Cannot capture multimodal posteriors; extension to exponential families is straightforward
\item \textbf{Quasi-static precision}: Full theory couples mean and precision dynamics
\item \textbf{Empirical validation}: Direct observation of underdamped oscillation remains for future work
\end{itemize}

%==============================================================================
\section{Conclusion}
%==============================================================================

We have demonstrated that beliefs naturally possess inertia proportional to prior precision, and that this single insight unifies major classical models from sociology, psychology, and network science. DeGroot social learning, Friedkin-Johnsen dynamics, bounded confidence, echo chambers, confirmation bias, Social Impact Theory, diffusion of innovations, collective intentionality, and Kuhnian paradigm shifts all emerge as limiting cases or regimes of Hamiltonian Variational Free Energy dynamics.

The key insight is that \emph{social structure creates epistemic consequences through information geometry}. Agents receiving high social attention develop inertial mass that resists belief revision---not through irrationality or motivated reasoning, but as geometric necessity on the statistical manifold.

This framework provides new tools for understanding persuasion, education, therapy, negotiation, and social dynamics. By recognizing that confident beliefs are massive and uncertain beliefs are light, and that the same equations govern individual cognition and collective opinion formation, we chart new frontiers in research and socio-psychological understanding.

The mathematics has been hiding in plain sight. The Fisher information metric has been whispering this entire time that it is an inertia tensor for the dynamics of thought.

\subsection*{Data Availability}
All code and data will be made publicly available upon publication at \\
https://github.com/cdenn016/Hamiltonian-VFE

\subsection*{Acknowledgments}
Claude Sonnet 4.5 was utilized for programming our variational free energy descent simulation suite. All code was manually reviewed, corrected, and mathematically validated by the author.

\appendix

\section{Gauge Frame Variations and Pullback Geometry}
\label{app:gauge}

\subsection{Gauge Structure of Multi-Agent Belief Systems}

The geometric setting is a principal $G$-bundle $\pi: P \to \mathcal{C}$ where $\mathcal{C}$ is the base manifold (agent positions, social network topology), $G = \mathrm{SO}(d)$ is the gauge group (rotations in belief space), and the fiber over each point is the space of reference frames.

Under gauge transformation $g_i \in \mathrm{SO}(d)$:
\begin{align}
\mu_i &\mapsto g_i \mu_i \\
\Sigma_i &\mapsto g_i \Sigma_i g_i^T \\
\Omega_{ik} &\mapsto g_i \Omega_{ik} g_k^{-1}
\end{align}

The full mean-sector mass matrix transforms as:
\begin{equation}
(\mathbf{M}^\mu)' = \mathbf{G} \, \mathbf{M}^\mu \, \mathbf{G}^T
\end{equation}

Hamilton's equations are fully covariant under these transformations, and physical observables (free energy, Hamiltonian, inter-agent divergences) are gauge-invariant.

\section{Hamiltonian Mechanics on Statistical Manifolds}
\label{app:hamiltonian}

\subsection{The Extended Free Energy Functional}

The complete variational free energy with explicit sensory evidence is:
\begin{equation}
\mathcal{F}[\{q_i\}] = \sum_i D_{\mathrm{KL}}(q_i \| p_i) + \sum_{i,k} \beta_{ik} D_{\mathrm{KL}}(q_i \| \Omega_{ik}[q_k]) - \sum_i \mathbb{E}_{q_i}[\log p(o_i \mid \theta)]
\end{equation}

\subsection{Complete Mass Matrix}

\paragraph{Mean sector diagonal:}
\begin{equation}
[\mathbf{M}^\mu]_{ii} = \bar{\Lambda}_{pi} + \sum_k \beta_{ik}\Omega_{ik}\Lambda_{qk}\Omega_{ik}^T + \sum_j \beta_{ji}\Lambda_{qi} + \Lambda_{o_i}
\end{equation}

\paragraph{Mean sector off-diagonal:}
\begin{equation}
[\mathbf{M}^\mu]_{ik} = -\beta_{ik}\Omega_{ik}\Lambda_{qk} - \beta_{ki}\Lambda_{qi}\Omega_{ki}^T \quad (i \neq k)
\end{equation}

\paragraph{Covariance sector diagonal:}
\begin{equation}
[\mathbf{M}^\Sigma]_{ii} = \frac{1}{2}(\Lambda_{qi} \otimes \Lambda_{qi}) \cdot \left(1 + \sum_k \beta_{ik} + \sum_j \beta_{ji}\right)
\end{equation}

\subsection{Hamilton's Equations}

\begin{align}
\dot{\mu}_i &= \sum_k [\mathbf{M}^{-1}]_{ik}^{\mu\mu}\pi_k^\mu \\
\dot{\pi}_i^\mu &= -\frac{\partial \mathcal{F}}{\partial \mu_i} - \frac{1}{2}\pi^T\frac{\partial \mathbf{M}^{-1}}{\partial \mu_i}\pi
\end{align}

The force decomposes into:
\begin{equation}
-\frac{\partial \mathcal{F}}{\partial \mu_i} = \underbrace{-\bar{\Lambda}_{pi}(\mu_i - \bar{\mu}_i)}_{\text{prior}} \underbrace{- \sum_k \beta_{ik}\tilde{\Lambda}_{qk}(\mu_i - \tilde{\mu}_k)}_{\text{consensus}} \underbrace{- \Lambda_{o_i}(\mu_i - o_i)}_{\text{sensory}}
\end{equation}

\begin{tcolorbox}[colback=gray!5,colframe=gray!75,title=Summary: The Complete Theory]

\textbf{State:} Each agent $i$ has belief $q_i = \mathcal{N}(\mu_i, \Sigma_i)$ with prior $p_i$ and observations $o_i$.

\textbf{Effective Mass:}
\begin{equation}
M_i = \bar{\Lambda}_{pi} + \sum_k \beta_{ik}\tilde{\Lambda}_{qk} + \sum_j \beta_{ji}\Lambda_{qi} + \Lambda_{o_i}
\end{equation}

\textbf{Physical Meaning:}
\begin{itemize}
    \item Position $\mu_i$ = what agent $i$ believes
    \item Momentum $\pi_i$ = rate of belief change $\times$ precision
    \item Mass = precision (tight beliefs are heavy)
    \item Force = pull toward prior + consensus + observations
\end{itemize}

\end{tcolorbox}

\bibliographystyle{apalike}
\bibliography{references}

\end{document}
